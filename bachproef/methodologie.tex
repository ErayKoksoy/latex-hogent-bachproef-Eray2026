%%=============================================================================
%% Methodologie
%%=============================================================================

\chapter{\IfLanguageName{dutch}{Methodologie}{Methodology}}%
\label{ch:methodologie}

%% TODO: In dit hoofstuk geef je een korte toelichting over hoe je te werk bent
%% gegaan. Verdeel je onderzoek in grote fasen, en licht in elke fase toe wat
%% de doelstelling was, welke deliverables daar uit gekomen zijn, en welke
%% onderzoeksmethoden je daarbij toegepast hebt. Verantwoord waarom je
%% op deze manier te werk gegaan bent.
%% 
%% Voorbeelden van zulke fasen zijn: literatuurstudie, opstellen van een
%% requirements-analyse, opstellen long-list (bij vergelijkende studie),
%% selectie van geschikte tools (bij vergelijkende studie, "short-list"),
%% opzetten testopstelling/PoC, uitvoeren testen en verzamelen
%% van resultaten, analyse van resultaten, ...
%%
%% !!!!! LET OP !!!!!
%%
%% Het is uitdrukkelijk NIET de bedoeling dat je het grootste deel van de corpus
%% van je bachelorproef in dit hoofstuk verwerkt! Dit hoofdstuk is eerder een
%% kort overzicht van je plan van aanpak.
%%
%% Maak voor elke fase (behalve het literatuuronderzoek) een NIEUW HOOFDSTUK aan
%% en geef het een gepaste titel.

Dit hoofdstuk beschrijft de aanpak van het onderzoek. De bachelorproef volgt een gefaseerde en iteratieve werkwijze waarbij de bestaande HOGENT-campusapplicatie wordt geanalyseerd, verbeterd en opnieuw geëvalueerd op vlak van digitale toegankelijkheid. Het onderzoek combineert theoretische onderbouwing met praktische implementatie en metingen.

\section{Literatuurstudie}

In de eerste fase wordt een literatuurstudie uitgevoerd om inzicht te verkrijgen in digitale toegankelijkheid binnen mobiele applicaties. Hierbij worden de WCAG 2.1 en 2.2 richtlijnen bestudeerd, met focus op niveau AA. Daarnaast wordt relevante Europese regelgeving onderzocht en worden wetenschappelijke publicaties geraadpleegd over toegankelijkheidsproblemen in mobiele toepassingen. Ook bestaande testtools zoals Accessibility Insights, Lighthouse en axe DevTools worden geanalyseerd. Deze fase resulteert in een overzicht van richtlijnen, best practices en evaluatiemethoden die als basis dienen voor de verdere uitvoering van het onderzoek.

\section{Toegankelijkheidsaudit van de huidige applicatie}

In deze fase wordt de huidige versie van de HOGENT-campusapplicatie systematisch geëvalueerd. De audit combineert geautomatiseerde testen met manuele controles. Geautomatiseerde tools detecteren technische problemen zoals contrastfouten en ontbrekende labels, terwijl manuele testen focussen op toetsenbordnavigatie en schermlezerondersteuning. De bevindingen worden gestructureerd volgens de WCAG-principes en vormen een nulmeting van de toegankelijkheidsstatus.

\section{Identificatie en prioritering van problemen}

Op basis van de auditresultaten worden de geïdentificeerde problemen geanalyseerd en geprioriteerd. Hierbij wordt rekening gehouden met de impact op gebruikers en de technische haalbaarheid van de oplossing. De problemen worden geclassificeerd per component en er wordt bepaald welke aanpassingen prioritair moeten worden uitgevoerd. Dit resulteert in een duidelijk actieplan voor implementatie.

\section{Implementatie van verbeteringen}

In deze fase worden de geselecteerde toegankelijkheidsproblemen technisch opgelost. Dit omvat onder meer het verbeteren van semantische HTML-structuren, het toepassen van ARIA-attributen waar nodig en het optimaliseren van kleurcontrasten en navigatie. De implementatie gebeurt iteratief, waarbij elke wijziging opnieuw getest wordt om regressie te vermijden. Het resultaat is een verbeterde versie van de applicatie waarin toegankelijkheid structureel is geïntegreerd.

\section{Evaluatie en vergelijking}

Na implementatie wordt de audit opnieuw uitgevoerd om de impact van de verbeteringen te meten. De resultaten worden vergeleken met de nulmeting om de vooruitgang objectief aan te tonen. Indien mogelijk worden aanvullende gebruikerstesten uitgevoerd om praktische feedback te verzamelen. Deze fase resulteert in een vergelijking tussen de oorspronkelijke en verbeterde situatie en vormt de basis voor de conclusies van het onderzoek.

Deze gefaseerde aanpak zorgt ervoor dat het onderzoek zowel theoretisch onderbouwd als praktisch gevalideerd is, en maakt het mogelijk om de onderzoeksvragen op een gestructureerde en meetbare manier te beantwoorden.
