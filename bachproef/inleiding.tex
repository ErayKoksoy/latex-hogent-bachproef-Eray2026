%%=============================================================================
%% Inleiding
%%=============================================================================

\chapter{\IfLanguageName{dutch}{Inleiding}{Introduction}}%
\label{ch:inleiding}

Mobiele applicaties spelen een steeds centralere rol binnen het hoger onderwijs. Ze ondersteunen studenten bij communicatie, studieplanning en het raadplegen van campusgerelateerde informatie \autocite{hinze2022}. Door deze digitalisering worden mobiele toepassingen een essentieel onderdeel van het dagelijkse studentenleven. Ondanks deze belangrijke functie worden digitale toegankelijkheidsrichtlijnen, zoals de Web Content Accessibility Guidelines (WCAG) \autocite{Vlaanderen_WCAG_nodate}, tijdens ontwikkeltrajecten nog te vaak onvoldoende geïntegreerd worden. Uit het \textcite{BelgianWebAccessibility2024} blijkt dat ongeveer 72\% van de Belgische overheidswebsites niet conform is aan de geldende toegankelijkheidsnormen. Aangezien onderwijsinstellingen deel uitmaken van de publieke sector, is het aannemelijk dat ook een deel van hun digitale toepassingen onvoldoende toegankelijk is. Dit vormt een fundamenteel probleem, aangezien ontoegankelijke digitale platformen een directe impact hebben op gelijke onderwijskansen.

\section{\IfLanguageName{dutch}{Probleemstelling}{Problem Statement}}%
\label{sec:probleemstelling}

Ondanks het groeiende belang van mobiele applicaties binnen het hoger onderwijs wordt digitale toegankelijkheid nog onvoldoende systematisch geïntegreerd in het ontwikkelproces. Richtlijnen zoals de WCAG bestaan, maar worden in de praktijk niet altijd consequent toegepast op mobiele toepassingen. Hierdoor ontstaan applicaties die wel functioneel zijn voor de gemiddelde gebruiker, maar drempels creëren voor studenten met een visuele, auditieve, motorische of cognitieve beperking.

Dit is problematisch, aangezien mobiele applicaties een essentieel onderdeel vormen van communicatie, studieorganisatie en toegang tot onderwijsinformatie. Ontoegankelijke toepassingen kunnen leiden tot verminderde zelfstandigheid, extra studiedruk en ongelijke onderwijskansen.

Deze bachelorproef richt zich concreet op de mobiele applicatie van HOGENT en heeft als doelgroep:
\begin{itemize}
    \item Het interne developmentteam en de IT-verantwoordelijken;
    \item De dienst studentenbegeleiding en inclusie;
    \item Studenten met een functiebeperking die de applicatie gebruiken.
\end{itemize}

\section{\IfLanguageName{dutch}{Onderzoeksvraag}{Research question}}%
\label{sec:onderzoeksvraag}

Wees zo concreet mogelijk bij het formuleren van je onderzoeksvraag. Een onderzoeksvraag is trouwens iets waar nog niemand op dit moment een antwoord heeft (voor zover je kan nagaan). Het opzoeken van bestaande informatie (bv. ``welke tools bestaan er voor deze toepassing?'') is dus geen onderzoeksvraag. Je kan de onderzoeksvraag verder specifiëren in deelvragen. Bv.~als je onderzoek gaat over performantiemetingen, dan 

\section{\IfLanguageName{dutch}{Onderzoeksdoelstelling}{Research objective}}%
\label{sec:onderzoeksdoelstelling}

Wat is het beoogde resultaat van je bachelorproef? Wat zijn de criteria voor succes? Beschrijf die zo concreet mogelijk. Gaat het bv.\ om een proof-of-concept, een prototype, een verslag met aanbevelingen, een vergelijkende studie, enz.

\section{\IfLanguageName{dutch}{Opzet van deze bachelorproef}{Structure of this bachelor thesis}}%
\label{sec:opzet-bachelorproef}

% Het is gebruikelijk aan het einde van de inleiding een overzicht te
% geven van de opbouw van de rest van de tekst. Deze sectie bevat al een aanzet
% die je kan aanvullen/aanpassen in functie van je eigen tekst.

De rest van deze bachelorproef is als volgt opgebouwd:

In Hoofdstuk~\ref{ch:stand-van-zaken} wordt een overzicht gegeven van de stand van zaken binnen het onderzoeksdomein, op basis van een literatuurstudie.

In Hoofdstuk~\ref{ch:methodologie} wordt de methodologie toegelicht en worden de gebruikte onderzoekstechnieken besproken om een antwoord te kunnen formuleren op de onderzoeksvragen.

% TODO: Vul hier aan voor je eigen hoofstukken, één of twee zinnen per hoofdstuk

In Hoofdstuk~\ref{ch:conclusie}, tenslotte, wordt de conclusie gegeven en een antwoord geformuleerd op de onderzoeksvragen. Daarbij wordt ook een aanzet gegeven voor toekomstig onderzoek binnen dit domein.