%%=============================================================================
%% Inleiding
%%=============================================================================

\chapter{\IfLanguageName{dutch}{Inleiding}{Introduction}}%
\label{ch:inleiding}

Mobiele applicaties spelen een steeds centralere rol binnen het hoger onderwijs. Ze ondersteunen studenten bij communicatie, studieplanning en het raadplegen van campusgerelateerde informatie \autocite{hinze2022}. Door deze digitalisering worden mobiele toepassingen een essentieel onderdeel van het dagelijkse studentenleven. Ondanks deze belangrijke functie worden digitale toegankelijkheidsrichtlijnen, zoals de Web Content Accessibility Guidelines (WCAG) \autocite{Vlaanderen_WCAG_nodate}, tijdens ontwikkeltrajecten nog te vaak onvoldoende geïntegreerd worden. Uit het \textcite{BelgianWebAccessibility2024} blijkt dat ongeveer 72\% van de Belgische overheidswebsites niet conform is aan de geldende toegankelijkheidsnormen. Aangezien onderwijsinstellingen deel uitmaken van de publieke sector, is het aannemelijk dat ook een deel van hun digitale toepassingen onvoldoende toegankelijk is. Dit vormt een fundamenteel probleem, aangezien ontoegankelijke digitale platformen een directe impact hebben op gelijke onderwijskansen.

\section{\IfLanguageName{dutch}{Probleemstelling}{Problem Statement}}%
\label{sec:probleemstelling}

Ondanks het groeiende belang van mobiele applicaties binnen het hoger onderwijs wordt digitale toegankelijkheid nog onvoldoende systematisch geïntegreerd in het ontwikkelproces. Richtlijnen zoals de WCAG bestaan, maar worden in de praktijk niet altijd consequent toegepast op mobiele toepassingen. Hierdoor ontstaan applicaties die wel functioneel zijn voor de gemiddelde gebruiker, maar drempels creëren voor studenten met een visuele, auditieve, motorische of cognitieve beperking.

Dit is problematisch, aangezien mobiele applicaties een essentieel onderdeel vormen van communicatie, studieorganisatie en toegang tot onderwijsinformatie. Ontoegankelijke toepassingen kunnen leiden tot verminderde zelfstandigheid, extra studiedruk en ongelijke onderwijskansen.

Deze bachelorproef richt zich concreet op de mobiele applicatie van HOGENT en heeft als doelgroep:
\begin{itemize}
    \item Het interne developmentteam en de IT-verantwoordelijken;
    \item De dienst studentenbegeleiding en inclusie;
    \item Studenten met een functiebeperking die de applicatie gebruiken.
\end{itemize}

\section{\IfLanguageName{dutch}{Onderzoeksvraag}{Research question}}%
\label{sec:onderzoeksvraag}

Uit de huidige problematiek blijkt dat de HOGENT-campusapp voor studenten met een beperking nog niet volledig toegankelijk is. Hoewel richtlijnen zoals WCAG bestaan, worden bepaalde toegankelijkheidsaspecten onvoldoende toegepast, waardoor sommige gebruikersgroepen niet optimaal van de app kunnen profiteren.  

Dit leidt tot de centrale onderzoeksvraag van deze bachelorproef:  
\textbf{Hoe kan de HOGENT-campusapp inclusiever worden ontworpen voor studenten met een beperking, rekening houdend met hun specifieke noden op het vlak van digitale toegankelijkheid?}

Deze hoofdvraag wordt verder opgesplitst in de volgende deelvragen:

\textbf{Probleemdomein:}
\begin{itemize}
    \item In welke mate voldoet de huidige versie van de HOGENT-campusapplicatie (POC) aan de WCAG-richtlijnen?
    \item Welke front-end componenten zijn momenteel ontoegankelijk volgens de WCAG-richtlijnen?
\end{itemize}

\textbf{Oplossingsdomein:}
\begin{itemize}
    \item Welke toegankelijkheidsaanpassingen kunnen op UI- en componentniveau technisch worden doorgevoerd om aan deze richtlijnen te voldoen?
    \item Welke meetbare verschillen in performantie treden op vóór en na de implementatie van toegankelijkheidsfunctionaliteiten, zowel aan front-end als aan back-end zijde?
    \item Welke testmethoden en evaluatiecriteria zijn geschikt om zowel toegankelijkheid als performantie van de applicatie objectief te meten?
    \item Hoe ervaren studenten met een beperking de bruikbaarheid en toegankelijkheid van de aangepaste campusapp in vergelijking met de oorspronkelijke versie?
\end{itemize}

\section{\IfLanguageName{dutch}{Onderzoeksdoelstelling}{Research objective}}%
\label{sec:onderzoeksdoelstelling}

Wat is het beoogde resultaat van je bachelorproef? Wat zijn de criteria voor succes? Beschrijf die zo concreet mogelijk. Gaat het bv.\ om een proof-of-concept, een prototype, een verslag met aanbevelingen, een vergelijkende studie, enz.

\section{\IfLanguageName{dutch}{Opzet van deze bachelorproef}{Structure of this bachelor thesis}}%
\label{sec:opzet-bachelorproef}

% Het is gebruikelijk aan het einde van de inleiding een overzicht te
% geven van de opbouw van de rest van de tekst. Deze sectie bevat al een aanzet
% die je kan aanvullen/aanpassen in functie van je eigen tekst.

De rest van deze bachelorproef is als volgt opgebouwd:

In Hoofdstuk~\ref{ch:stand-van-zaken} wordt een overzicht gegeven van de stand van zaken binnen het onderzoeksdomein, op basis van een literatuurstudie.

In Hoofdstuk~\ref{ch:methodologie} wordt de methodologie toegelicht en worden de gebruikte onderzoekstechnieken besproken om een antwoord te kunnen formuleren op de onderzoeksvragen.

% TODO: Vul hier aan voor je eigen hoofstukken, één of twee zinnen per hoofdstuk

In Hoofdstuk~\ref{ch:conclusie}, tenslotte, wordt de conclusie gegeven en een antwoord geformuleerd op de onderzoeksvragen. Daarbij wordt ook een aanzet gegeven voor toekomstig onderzoek binnen dit domein.